 % !TEX root = report.tex
 % !TEX program = xelatex

\subsection{项目背景}
Console,中文名为控制台,是一种计算机的物理设备。它只具有基本的I/O功能,而不具有计算能力。
在计算机刚诞生时,控制台是与计算机独立的。它一般具有键盘与屏幕,将用户输入传送给计算机,将计算机的输出回显给用户。

随着计算机的发展,越来越少的计算机采取这种分离的设计,操作系统都会内置虚拟控制台(Virtual Console),通过软件即可模拟出控制台的功能。
实现这类技术的软件被称为终端模拟器(Terminal Emulator),如Linux下的\texttt{xterm}, \texttt{gnome-terminal}, \texttt{Konsole}, \texttt{screen},Windows下的\texttt{cmd.exe}、超级终端等工具都属于终端模拟器。
另外,借助于称为“虚拟终端”(Virtual Terminal)的技术,也不再有一台计算机只能由一个用户使用的限制。

\subsection{功能与特性}

本项目实现的是一个真正的\textbf{物理}终端!它可以通过串口与计算机连接,通过PS/2接口连接键盘,HDMI接口连接显示器。
之后,在PC上透过串口运行\texttt{getty}等终端管理软件启动一个Shell,就可以像使用任何一个熟悉的终端模拟器一样使用它。

为了正确地交换数据,终端与PC通信遵循不同的规范,它们规定了两端互相发送数据的格式。
从PC到终端,有字符编码、光标控制、字符颜色、插入模式等信息;从PC到终端,每一个(或者一组)按键都对应着不同的一串字符串。
我们完整地实现了被接受最为广泛的\texttt{VT220}标准\footnote{介绍可见 \url{https://en.wikipedia.org/wiki/VT220},以及下面给出的技术文档},它在 Linux、macOS 等操作系统中都有完善的支持。
鉴于\texttt{VT220}的显示只是单色模式,为了更好的效果,我们还实现了\texttt{xterm-256color}\footnote{介绍可见 \url{https://en.wikipedia.org/wiki/Xterm}}模式下的大部分影响显示效果的指令,尤其是增加了对颜色的支持。
我们实现了标准中的所有颜色模式与字符特效,遗憾的是受到实验板的限制,最终的输出会被化归为512色。

本项目可以与运行任何现代操作系统的PC配合,运行日常所用的命令行模式的软件,也可以用来连接网络设备等进行配置、调试工作。

本项目全部采用\texttt{SystemVerilog}进行编写。它带来了多个方便的新特性,如:
\begin{description}
  \item[\texttt{logic}类型] 整合了 \texttt{wire} 和 \texttt{reg} 类型,自动推导几类寄存器模型,减少无谓的区分。
  \item[\texttt{always\_comb, always\_latch, always\_ff}块] 可明确指定需要的逻辑类型(纯组合逻辑、锁存器、触发器),防止出现非预期电路和行为。
  \item[\texttt{struct}类型] 与C语言类似,可将一组值/信号进行单独赋值、统一绑定与传递。
  \item[\texttt{enum}类型] 与C语言类似,可以给常量赋名称,并有作用域限制,在状态机的设计中减少了出错的可能。
  \item[\texttt{typedef}关键字] 与C语言类似,可以给复杂的变量类型赋予别名,更直观好记。
  \item[\texttt{define}与\texttt{generate}关键字] 可以方便地在预处理阶段批量生成代码,以及在综合阶段批量生成元件,适合大量小元器件并行使用的场合。
\end{description}

进一步地,\texttt{SystemVerilog}在电路的测试验证中具有 \texttt{VHDL} 与 \texttt{Verilog}不可比拟的灵活性、便捷性。我们强烈建议各位学习这门先进的硬件描述语言,也建议老师在今后的教学中考虑采纳它,与现代VLSI设计接轨。

另外,本项目还有以下的突出特性:
\begin{description}
    \item[高分辨率] 我们实现了800*600@72Hz的VGA信号输出,一屏幕能容纳100列*50行=5000个文字,相比标准要求的80列*25行=2000字提升了$250\%$。
    \item[高速度] 通过对时序的精心优化,我们的解析与渲染模块都做到了全流水线运行,最高能运行在 120 MHz 的频率下。
    因此,我们能以3M的波特率与计算机串口进行通信,从而能够以25FPS流畅播放电影(由播放器渲染为动态彩色文字输出)。
    \item[高稳定性] 我们充分发掘SRAM潜力,在渲染与显示器输出中采用了双缓冲机制,解决了屏幕撕裂、像素微小抖动等常见问题。
    \item[高模块化] 我们在整个工程中,坚持高内聚低耦合的思想,将每个重要功能都抽取为独立的模块,减少代码冗余,也增强了可扩展性。同时VGA、显存、键盘等模块通用性强,可用在今后的其他项目中。
\end{description}


\subsection{项目分工}
\begin{minipage}[t]{0.45\textwidth}
陈晟祺:
\begin{itemize}
    \item 项目总体架构
    \item 键盘控制模块(PS2、译码)
    \item 显示控制模块(VGA、SRAM)
\end{itemize}
\end{minipage}
\begin{minipage}[t]{0.45\textwidth}
周聿浩:
\begin{itemize}
    \item 终端命令解析与处理
    \item 项目综合调试
    \item 展示Demo设计
\end{itemize}
\end{minipage}

\vspace{1em}
注:各模块的详细功能与设计可见下面各节,两人的工作量基本是均衡的。