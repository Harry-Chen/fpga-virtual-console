\subsubsection[显示控制器]{显示控制器 \texttt{DisplayController}}

显示控制器的主要功能由两个模块实现:\texttt{TextRenderer}负责读取RAM中存储的文本,渲染为像素数据并存储在SRAM的显存中;\texttt{VgaDisplayAdapter}负责产生VGA信号,从SRAM显存中读取像素数据并输出。各部分的详细实现如下。

\paragraph[SRAM控制器]{SRAM控制器\texttt{SramController}}
由于分辨率较高

\begin{table}[htbp]
\centering
    \caption{SRAM中每条数据的存储结构}
    \label{tab:sram_bytefield}
    \vspace{1em}
    \begin{bytefield}[endianness=big,boxformatting={\centering\tt}]{32}
        \bitheader{0,9,18,31} \\
        \bitbox{14}{Empty} & \bitbox{9}{Even Pixel} &
        \bitbox{9}{Odd Pixel}
    \end{bytefield}
\end{table}


\paragraph[字符渲染器]{文本渲染器\texttt{TextRenderer}}
文本渲染器是一个状态机,工作如下:
\begin{enumerate}
  \item 从RAM中读取下一行文本(可循环)
  \item 如果未到行末,读取下一个字符,并从\texttt{FontRom}中读取相应字形;否则跳回1
  \item 将字形、属性(如颜色、特效等)和当前字符第一个像素在显存中的地址传递给内部的子模块\texttt{FontShapeRenderer},指示开始渲染
  \item \texttt{FontShapeRenderer}渲染完成后,跳回到2
\end{enumerate}

其中\texttt{FontRom}是一个存储了ASCII中256个字符对应的形状的ROM,每个字符宽8像素,高12像素,地址即为自身的ASCII编码。\texttt{FontShapeRenderer}是用于渲染一个字的子模块,也是一个状态机,工作如下:
\begin{enumerate}
  \item 等待父渲染器的开始信号,并接受传递的信息
  \item 如果没有到最后一个像素,读取下一个像素,否则报告完成并回到1
  \item 锁存上一个像素的颜色数据,确定下一个像素的颜色,计算这两个像素在显存中的具体地址,向SRAM控制器发出写请求
  \item 如果控制器报告写请求完成,跳到2,否则等待
\end{enumerate}


\paragraph[VGA控制器]{VGA控制器\texttt{VgaDisplayAdapter}}


