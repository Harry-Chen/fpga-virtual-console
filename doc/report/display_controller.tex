 % !TEX root = report.tex
 % !TEX program = xelatex

\subsection{显示控制器 \texttt{DisplayController}}

显示控制器的主要功能由三个模块实现:\texttt{SramController}负责控制SRAM的读写,\texttt{TextRenderer}负责读取RAM中存储的文本,渲染为像素数据并存储在SRAM的显存中;\texttt{VgaDisplayAdapter}负责产生VGA信号,从SRAM显存中读取像素数据并输出。各部分的详细实现如下。

\subsubsection{SRAM控制器 \texttt{SramController}}
我们使用片外的SRAM作为显存,并且采用了双缓冲机制,即使用两块显存A与B,当渲染器向A写入时,显示模块从B进行读取,当二者均完成后进行交换,如此循环。这样保证了渲染和输出过程都有一致性,不会产生画面撕裂、闪烁等异常现象。
由于使用的SRAM读写速度均在20ns左右,因此SRAM控制器最高时钟频率为50MHz,这就导致了读/写实际的时钟频率只有25MHz。
而由于我们采用的VGA时序要求的像素频率为50MHz,因此我们必须至少在一次读写中操作相邻的两个像素,才能保证输出图像是正常的。
因此,我们设计了表\ref{tab:sram_bytefield}中所示的SRAM存储结构。对于屏幕上同一行上相邻的$(x,y)$与$(x,y+1)$两个像素,它们相对于当前显存基址的偏移都是$\frac{x+y}{2}$。

\begin{table}[htbp]
\centering
    \caption{SRAM中每条数据的存储结构}
    \label{tab:sram_bytefield}
    \vspace{1em}
    \begin{bytefield}[endianness=big,boxformatting={\centering\tt}]{32}
        \bitheader{0,9,18,31} \\
        \bitbox{14}{Empty} & \bitbox{9}{Even Pixel} &
        \bitbox{9}{Odd Pixel}
    \end{bytefield}
\end{table}

由于VGA模块的读取请求和渲染器的写入请求并非总是间隔进行(如VGA消隐区时不进行读取,渲染器读取文本数据时不进行写入),因此简单的Round-Robin算法不能满足我们的需要。
而又因为渲染模块是由状态机驱动,可以任意进行等待;VGA模块的读取请求如果失败,就会出现渲染问题。
为此,我们为SRAM控制器设计了优先机制,即其在每次时钟沿时优先处理VGA模块的读取请求,如果没有,再处理渲染模块的写入请求。
测试证明,这样的机制能够充分利用每一个时钟周期,没有浪费,也保证了渲染与显示模块的正常工作。

需要特别注意的是,正如往年同学在实验报告中指出的,我们原本使用的实验板的SRAM请求都经过了控制FPGA的转发,读写延时都比预期要长。我们实测,这样会导致控制器无法运行在50MHz的频率下。
为了减少不必要的麻烦(如重写烧写控制FPGA,或者在SRAM一条记录中存储更多像素),我们更换了另一块FPGA直接连接到SRAM的实验板,避免了这一问题的发生。

\subsubsection{文本渲染模块 \texttt{TextRenderer}}
文本渲染模块的实现是一个状态机,工作如下:
\begin{enumerate}
  \item 从RAM中读取下一行文本(可循环)
  \item 如果未到行末,读取下一个字符,并从\texttt{FontRom}中读取相应字形;否则跳回1
  \item 将字形、属性(如颜色、特效等)和当前字符第一个像素在显存中的地址传递给内部的子模块\texttt{FontShapeRenderer},指示开始渲染
  \item \texttt{FontShapeRenderer}渲染完成后,跳回到2
\end{enumerate}

其中\texttt{FontRom}是一个存储了ASCII中256个字符对应的形状的ROM,每个字符宽8像素,高12像素,地址即为自身的ASCII编码。\texttt{FontShapeRenderer}是用于渲染一个字的子模块,也是一个状态机,工作如下:
\begin{enumerate}
  \item 等待父渲染器的开始信号,并接受传递的信息
  \item 如果未到最后一个像素,计算下一个像素的颜色并锁存数据;否则报告完成并回到1
  \item 计算下一个像素的颜色,计算这两个像素在显存中的具体地址,时钟沿末向SRAM控制器发出写请求
  \item 如果控制器报告写请求完成,跳到2,否则等待
\end{enumerate}


\subsubsection{VGA显示模块 \texttt{VgaDisplayAdapter}}
VGA控制器负责产生VGA图形信号(我们使用的实验板上实际是HDMI接口,不过我们需要给出的信号格式是一致的)。为了使我们的终端实用性更强,我们实现了800*600输出分辨率,也即50行*100列,每一个单元格一个字符。
为了配合现有的SRAM等模块工作,结合 \url{http://tinyvga.com/vga-timing} 给出的VGA时序规范,我们选取了72Hz刷新率,使得输出的像素频率刚好为50MHz,具体的时序可见表\ref{tab:vga_timing}。

\begin{table}[htbp]
\centering
\caption{800*600@72Hz的VGA信号时序规范}
\label{tab:vga_timing}
\begin{tabular}{@{}|c|c|c||c|c|c|@{}}
  \hline
  \textbf{Scanline part}                       & \textbf{Pixels} & \textbf{Time ($\mu$s)} & \textbf{Frame part} & \textbf{Lines} & \textbf{Time (ms)} \\ \hline
  Visible area                                 & 800             & 16                     & Visible area        & 600            & 12.48              \\ 
  Front porch                                  & 56              & 1.12                   & Front porch         & 37             & 0.7696             \\ 
  Sync pulse                                   & 120             & 2.4                    & Sync pulse          & 6              & 0.1248             \\ 
  Back porch                                   & 64              & 1.28                   & Back porch          & 23             & 0.4784             \\ \hline
  Whole line                                   & 1040            & 20.8                   & Whole frame         & 666            & 13.8528            \\ \hline
\end{tabular}
\end{table}

由于需要预读取像素数据,VGA控制器也被设计为一个状态机。VGA的时序信号发生模块是独立于状态机工作的,保证信号严格遵守规范输出。状态机的工作如下:
\begin{enumerate}
  \item 重置时序发生模块,时钟沿末发送对最早两个像素的读取请求
  \item 锁存得到的两个像素数据,时钟沿末输出奇数像素
  \item 时钟沿末输出偶数像素,如果下一个像素依旧在图像区域,则计算出需要读取的内存地址并在时钟沿末发送读取请求,跳转到2;否则跳转到4
  \item 不断等待,直到下一个像素回到图像区域,计算出需要读取的内存地址并发送读取请求,跳转到2
\end{enumerate}
  